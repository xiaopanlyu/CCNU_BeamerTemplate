%# -*- coding:utf-8 -*-
%\documentclass[10pt,aspectratio=169,mathserif]{beamer}	
\documentclass[12pt,aspectratio=169,compress,t,noamsthm,notheorem,handout,table]{ctexbeamer}	
%设置为 Beamer 文档类型,设置字体为 10pt,长宽比为16:9,数学字体为 serif 风格

%%%%-----导入宏包-----%%%%
\usepackage{ccnu_nercl}			%导入 CCNU 模板宏包
\usepackage{ctex}			%导入 ctex 宏包,添加中文支持
\usepackage{amsmath,amsfonts,amssymb,bm}   %导入数学公式所需宏包
\usepackage{color}			 %字体颜色支持
\usepackage{graphicx,hyperref,url}
\usepackage{metalogo}	% 非必须
%% 上文引用的包可按实际情况自行增删
%%%%%%%%%%%%%%%%%%	


%\beamertemplateballitem		%设置 Beamer 主题

%%%%------------------------%%%%%
\catcode`\。=\active         %或者=13
\newcommand{。}{.}				
%将正文中的“。”号转换为“.”。中文标点国家规范建议科技文献中的句号用圆点替代
%%%%%%%%%%%%%%%%%%%%%


%begin>>>>set title page>>>>>>>>>>>>>>>>>>>>>>>>>>>>>>>>>>>>>>>>>>>>>>>>>>
\title[基于深度网络提取题目中直陈数学关系的研究]{初等数学代数型题目机器解答研究}%title
\subtitle{——基于深度网络提取题目中直陈数学关系的研究}			
\author[XIAOPAN LYU]{%个人信息设置
  吕小攀, a 1st year Ph.D. student. \\
  {\url{xiaopanlyu@mails.ccnu.edu.cn}} \\
  {\url{https://github.com/xiaopanlyu}}
}
\institute[NERCL]{%机构信息
   华中师范大学$\bullet$ 国家数字化学习工程技术研究中心
}
\date[\today]{\today}%日期信息
%end>>>>首页信息设置>>>>>>>>>>>>>>>>>>>>>>>>>>>>>>>>>>>>>>>>>>>>>>>>>>>>>>>>
  
\begin{document}


%----------------title page---------------
\begin{frame}[plain]
	\titlepage
\end{frame}

%----------------Outline page---------------
\begin{frame}
	\frametitle{Outline}
 	\tableofcontents[hideallsubsections]
\end{frame}


%----------------insert outline at the begining of each section---------------
\AtBeginSection[]{\frame{\tableofcontents[currentsection,hideallsubsections]}}

\section{Introduction}
\begin{frame}
  \frametitle{介绍}

  \begin{itemize}
    \item {编译方式}
	    \begin{itemize}
	    	\item  推荐安装完整版的 TeXLive
	    	\item 使用 \XeLaTeX 编译
	    \end{itemize}
    \item 请参考 \LaTeX 和 Beamer 用户文档 
    
    \item 行内数学公式示例 $\sin^2 \theta + \cos^2 \theta = 1$
    \item {行间数学公式示例 \begin{equation}
	    y_{1}=\int \sin x\, {\rm d}x
    \end{equation}	 }   
    \item 基于“华大绿”颜色 \url{http://www.ccnu.edu.cn/}
  \end{itemize}
\end{frame}

\section{内置环境}
\begin{frame}
  \frametitle{内置环境}
	\begin{block}{Slides with \LaTeX}
	    Beamer offers a lot of functions to create nice slides using \LaTeX.
	  \end{block}
	
	  \begin{block}{The basis}
	    内部使用以下主题
	    \begin{itemize}
	      \item split
	      \item whale
	      \item rounded
	      \item orchid
	    \end{itemize}
	  \end{block}
\end{frame}

\begin{frame}
  \frametitle{带数字列表}
	 \begin{enumerate}
	    \item This just shows the effect of the style
	    \item It is not a Beamer tutorial
	    \item Read the Beamer manual for more help
	    \item Contact me only concerning the style file
	  \end{enumerate}
\end{frame}

\section{结论}
\begin{frame}
  \frametitle{结论}

  \begin{itemize}
    \item Easy to use
    \item Good results
  \end{itemize}
\end{frame}

\section{参考文献}
\begin{frame}{参考文献}
\begin{thebibliography}{99} 
\bibitem{zhao1} Yi~Zhao, {\sl An introduction to X}, Sep.~15, 2015
\bibitem{qian2} Er~Qian, San~Sun, 
Phys.\ Lett.\ A {\bf xx}, 2xx (20xx)   
\bibitem{li4} Si~Li, Phys.\ Rev.\ C {\bf xx}, 5xx (20xx) 

\end{thebibliography}
\end{frame}

\end{document}
